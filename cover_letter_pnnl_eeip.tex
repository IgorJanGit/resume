\documentclass[letterpaper,11pt]{article}
\usepackage[empty]{fullpage}
\usepackage[hidelinks]{hyperref}
\usepackage[english]{babel}
\usepackage{parskip}
\usepackage{titlesec}
\usepackage{tabularx}

\titleformat{\section}{\large\bfseries}{}{0em}{}

\begin{document}

\begin{center}
  {\Large Igor Janotti}\\
  (425) 638--3519 $\mid$ \href{mailto:IgorJ1@live.com}{IgorJ1@live.com} $\mid$ \href{https://github.com/IgorJanGit}{github.com/IgorJanGit}
\end{center}

\vspace{0.5em}
\noindent February 1, 2026\\
Electricity Infrastructure \& Buildings Division\\
Pacific Northwest National Laboratory (PNNL)\\
Richland, WA

\vspace{1em}
\noindent \textbf{Nuclear Sciences EEIP Cover Letter}

\vspace{0.5em}
\noindent Dear EEIP Hiring Committee,

\vspace{0.5em}
I am excited to apply for the EEIP Undergrad Intern \textemdash{} Electricity Infrastructure \& Buildings Research (Job ID 11240). I am motivated by PNNL's mission to accelerate a resilient, efficient energy system and I am eager to contribute to research in grid modernization, building systems, and energy analytics.

At Bellevue College, my Computer Science coursework in Software Engineering, Cloud Computing, Data Structures, Algorithms, and Database Systems has provided a strong foundation in building reliable, scalable software and data pipelines. Beyond the classroom, I contributed to \emph{Eclipse VOLTTRON}, an agent-based platform for distributed energy and building systems, developing a BACnet scanner to improve device discovery and an AI-driven CLI to streamline developer interaction. I also supported testing and CI workflows for the Open Energy Dashboard and served as a Teaching Assistant for C++, reinforcing core engineering practices and collaborative problem-solving.

My technical experience spans Python, Java, C/C++, and C\#, with practical exposure to Docker, AWS, GitHub Actions, and event-driven architectures. I enjoy working at the intersection of software and energy systems, and I am particularly interested in projects involving:\\
\begin{itemize}
  \item Power Systems Engineering: architecture, modeling/simulation, and secure communications
  \item Energy System Analytics \& Control: learning/predictive analytics, optimization, sensing, and measurement
  \item Building System Sciences \& Engineering: building energy modeling, electrical integration, and performance evaluation
  \item Policy, Markets \& Deployment: data-driven analysis and valuation supporting deployment decisions
\end{itemize}

I am available to begin with the Summer 2026 cohort on either May 26, 2026 or June 16, 2026, and I am open to Richland, Seattle, or Portland locations. I value the opportunity to learn from PNNL researchers, contribute to high-impact projects, and grow through The Gold Experience.

Thank you for your time and consideration. I would be grateful for the chance to further discuss how my background and project experience can support EEIP research priorities.

\vspace{1em}
Sincerely,\\
Igor Janotti

\vspace{1em}
\noindent Link to posting: \href{https://careers.pnnl.gov/jobs/11240?lang=en-us}{careers.pnnl.gov/jobs/11240}

\end{document}
