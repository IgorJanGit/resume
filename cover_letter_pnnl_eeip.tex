\documentclass[letterpaper,11pt]{article}
\usepackage[empty]{fullpage}
\usepackage[hidelinks]{hyperref}
\usepackage[english]{babel}
\usepackage{parskip}
\usepackage{titlesec}
\usepackage{tabularx}

\titleformat{\section}{\large\bfseries}{}{0em}{}

\begin{document}

\begin{center}
  {\Large Igor Janotti}\\
  (425) 638--3519 $\mid$ \href{mailto:IgorJ1@live.com}{IgorJ1@live.com} $\mid$ \href{mailto:Igor.janotti@bellevuecollege.edu}{Igor.janotti@bellevuecollege.edu} $\mid$ \href{https://github.com/IgorJanGit}{github.com/IgorJanGit}
\end{center}

\vspace{0.5em}
\noindent February 1, 2026\\
Electricity Infrastructure \& Buildings Division\\
Pacific Northwest National Laboratory (PNNL)\\
Richland, WA

\vspace{1em}
\noindent \textbf{Nuclear Sciences EEIP Cover Letter}

\vspace{0.5em}
\noindent Dear EEIP Hiring Committee,

\vspace{0.5em}
I am applying for the EEIP Undergrad Intern \textemdash{} Electricity Infrastructure \& Buildings (Job ID 11240). I meet EEIP eligibility as a degree-seeking undergraduate (GPA 3.54, 6+ credits) and am available to start on May 26 or June 16, 2026. I’m open to Richland, Seattle, or Portland.

As a dedicated student who completed my associate degree and am now pursuing Computer Science, I thrive on challenges and continuous learning. I have a passion for organization, projects, and problem-solving, and I’m a positive, team-oriented, and motivated individual. My experience spans Microsoft Word, Excel, PowerPoint, Outlook, and programming with Java, Python, C#, C/C++, object-oriented design, MySQL, MongoDB, and ARM assembly. These experiences strengthened communication, resourcefulness, adaptability, and the ability to work effectively across cultures and diverse teams.

At Bellevue College, coursework in Software Engineering, Cloud Computing, Data Structures, Algorithms, and Database Systems prepared me to build reliable analytics and control software. On \emph{Eclipse VOLTTRON}, an agent-based distributed platform for building and energy systems, I implemented BACnet device discovery tooling and event-driven components that improved system observability and streamlined operator workflows. I also contributed automated testing and CI to the Open Energy Dashboard and served as a C++ Teaching Assistant, reinforcing disciplined engineering practices.

Skills snapshot: Python, Java, C/C++, C#; MySQL/SQL and MongoDB; protocols and platforms including BACnet, MQTT, and Eclipse VOLTTRON; tools such as Git, CI/CD, and automated testing. I work with distributed, agent-based, event-driven architectures for building-system integration at the grid edge, applying predictive analytics and optimization with attention to sensing, measurement, modeling/simulation, secure communications, and performance evaluation.

I’m particularly interested in the following EEIP areas:
\begin{itemize}
  \item Power Systems Engineering: architecture, modeling/simulation, and secure communications
  \item Energy System Analytics \& Control: learning/predictive analytics, optimization, sensing, and measurement
  \item Building System Sciences \& Engineering: building energy modeling, electrical integration, and performance evaluation
  \item Policy, Markets \& Deployment: data-driven analysis and valuation supporting deployment decisions
\end{itemize}

I value the opportunity to learn from PNNL researchers, contribute to high-impact projects, and grow through The Gold Experience.

Thank you for your time and consideration. I would be grateful for the chance to further discuss how my background and project experience can support EEIP research priorities.

\vspace{1em}
Sincerely,\\
Igor Janotti

\vspace{1em}
\noindent Link to posting: \href{https://careers.pnnl.gov/jobs/11240?lang=en-us}{careers.pnnl.gov/jobs/11240}

%
% Appended older letter block has been integrated into the narrative above and removed for concision.

\end{document}
